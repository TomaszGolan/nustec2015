\section{Tutorial MC}

\begin{slide}[method=direct]{PRNG}
\null\vfill

  \begin{itemize}
   \item You can use whatever random number generator you want
   \item If you are using C++ you may consider using PRNG class, which wraps up mersenne twister engine \href{https://raw.githubusercontent.com/TomaszGolan/simpleClassifiers/master/include/PRNG.h}{[link to PRNG.h]}
   \item Usage:
   
  \end{itemize}

  \begin{verbatim}
const PRNG random (min, max);
random.generate00(); // returns RN from (min, max)
random.generate01(); // returns RN from (min, max]
random.generate10(); // returns RN from [min, max)
random.generate11(); // returns RN from [min, max]
   \end{verbatim}

\vfill\null
\end{slide}

\begin{slide}[toc = Task 1]{Task 1: evaluate $\pi$}
\null\vfill

  \twocolumn
  {
    \sep\sep
    Evaluate $\pi$ using MC method
    \begin{itemize}
      \item get $N$ random points from a square
      \item count how many points are inside a circle
      \item calculate $\pi$
    \item check how the results depends on $N$
    \end{itemize}
  }
  {
    \usetikzlibrary{calc}

\begin{tikzpicture}

  \draw[>=latex, <->, thick] node[left, yshift = 4cm]{$y$} ++ (0, 4) -- (0, 0) -- node[below, xshift = 2cm]{$x$} ++ (4,0);
  
  \foreach \x in {1,...,200}
  {
    \pgfmathparse{rnd}
    \pgfmathsetmacro{\a}{2.9*\pgfmathresult + 0.05}
    \pgfmathparse{rnd}
    \pgfmathsetmacro{\b}{2.9*\pgfmathresult + 0.05}
    \pgfmathsetmacro{\c}{(\a - 1.5)*(\a - 1.5) + (\b - 1.5)*(\b - 1.5) - 1.5*1.5}
        
    \ifthenelse{\lengthtest{\c pt > 0.2 pt}}{\draw[filled, color=pdcolor7] (\a, \b) circle (0.03);}{}
    \ifthenelse{\lengthtest{\c pt < - 0.2 pt}}{\draw[filled, color=pdcolor6] (\a, \b) circle (0.03);}{}
  }

  \draw[ultra thick] (1.5, 1.5) circle (1.5);
  
  \draw[color=pdcolor3, dashed] node[left, yshift = 3cm]{a} ++ (0, 3) -- (3,3) -- node[yshift = -1.75cm]{a} (3, 0);
      
\end{tikzpicture}

  }

\vfill\null
\end{slide}

\begin{slide}[toc = Task 2]{Task 2: integration}
\null\vfill

  \twocolumn
  {
      Lets consider the following function:
  
      $$F(Q^2) = \frac{1}{(1 + Q^2)^2}$$
      
    \begin{itemize}

      \item[a)] Integrate this function over $Q^2$ using hit-or-miss method
      
      \item[b)] Integrate this function over $x = \mbox{log}_{10}(Q^2)$ using the same method
      
      \item[c)] Compare efficiency
      
      \item[d)] Integrate this function using crude method
      
    \end{itemize}    
  }
  {
    \usetikzlibrary{calc}

\pgfplotsset{every  tick/.style={pdcolor3,}, minor x tick num=1,}

\begin{tikzpicture}

  \begin{axis}[xlabel = $Q^2$, ylabel = $F(Q^2)$, ylabel near ticks, domain=1:10, scale=0.5, axis lines = left, inner axis line style={>=latex}, ymin = 0, ymax = 0.275, xmin = 1, xmax = 10.5]
    
    \addplot[mark=none, color=pdcolor1, ultra thick] {1 / (1 + x) / (1 + x)};
    \addplot[mark=none, dashed, color=pdcolor3, thick] coordinates {(1,0.25) (10,0.25) (10,0)};

    
    \foreach \x in {1,...,200}
    {
      \pgfmathparse{rnd}
      \pgfmathsetmacro{\a}{9.0*\pgfmathresult + 0.95}
      \pgfmathparse{rnd}
      \pgfmathsetmacro{\b}{0.23*\pgfmathresult + 0.01}
      \pgfmathsetmacro{\c}{\b - 1 / (1 + \a) / (1 + \a)}
      	  
      \ifthenelse{\lengthtest{\c pt > 0.01 pt}}{\addplot[mark=*, mark size = 1pt, color=pdcolor6] coordinates {(\a, \b)};}{}
      \ifthenelse{\lengthtest{\c pt < -0.01 pt}}{\addplot[mark=*, mark size = 1pt, color=pdcolor7] coordinates {(\a, \b)};}{}
    }
    
  \end{axis}

\end{tikzpicture}
    \vspace{-10pt}
    \usetikzlibrary{calc}

\pgfplotsset{every  tick/.style={pdcolor3,}, minor x tick num=1,}

\begin{tikzpicture}

  \begin{axis}[xlabel = {$x = \mbox{log}_{10}(Q^2)$}, ylabel = $F(x)$, ylabel near ticks, domain=-2:1, scale=0.5, axis lines = left, inner axis line style={>=latex}, ymin = 0, ymax = 1.25, xmin = -2, xmax = 1.25]
    
    \addplot[mark=none, color=pdcolor1, ultra thick] {1 / (1 + 10^x) / (1 + 10^x)};
    \addplot[mark=none, dashed, color=pdcolor3, thick] coordinates {(-2,1) (1,1) (1,0)};
    
    \foreach \x in {1,...,200}
    {
      \pgfmathparse{rnd}
      \pgfmathsetmacro{\a}{-2.90*\pgfmathresult + 0.95}
      \pgfmathparse{rnd}
      \pgfmathsetmacro{\b}{0.90*\pgfmathresult + 0.05}
      \pgfmathsetmacro{\d}{-0.175 * (\a + 2) * (\a + 2) + 1}
      \pgfmathsetmacro{\c}{\b - \d}
      	  
      \ifthenelse{\lengthtest{\c pt > 0.1 pt}}{\addplot[mark=*, mark size = 1pt, color=pdcolor6] coordinates {(\a, \b)};}{}
      \ifthenelse{\lengthtest{\c pt < - 0.1 pt}}{\addplot[mark=*, mark size = 1pt, color=pdcolor7] coordinates {(\a, \b)};}{}
    }
    
  \end{axis}

\end{tikzpicture}
    
  }


\vfill\null
\end{slide}

\begin{slide}[toc = Task 3]{Task 3: generating number from distribution}
\null\vfill

  \twocolumn
  {
    \sep\sep
  
      Write a program to generate random numbers from $[0,1]$ according to the following distribution:
      
      $$f(x) = x^3$$      
  }
  {
    \usetikzlibrary{calc}

\pgfplotsset{every  tick/.style={pdcolor3,}, minor x tick num=1,}

\begin{tikzpicture}

  \begin{axis}[xlabel = $x$, ylabel = $y$, ylabel near ticks, domain=0:1, scale=0.5, axis lines = left, inner axis line style={>=latex}, ymin = 0, ymax = 1.25, xmin = 0, xmax = 1.25]
    
    \addplot[mark=none, color=pdcolor1, ultra thick] {x^3};

  \end{axis}

\end{tikzpicture}
  }
  
  \begin{itemize}
      \item[a)] using cumulative distribution function
      
      \item[b)] using acceptance-rejection method (consider substitution to get better performance)
  \end{itemize}

\vfill\null
\end{slide}

\begin{slide}[toc = Task 4*]{Task 4*: neutrino-electron scattering}
\null\vfill

  \begin{itemize}
   \item For $E_\nu >> m_e$, the cross section for $\nu_\mu-e$ scattering can be approximated by:
   
   $$\frac{d\sigma}{dy} = \frac{G_F^2 s}{\pi}\left[A^2 + B^2\cdot(1 - y)^2\right]$$
   
   where $G_F$ - Fermi weak coupling constant, $s$ - Mandelstam variable, $y \equiv \frac{T_e}{E_\nu}$ with $T_e$ - electron kinetic energy, $A = \frac{1}{2} - \sin^2\theta_W$, $B = \sin^2\theta_W$, $\theta_W$ - Weinberg angle
   
   \item[a)] Write a program to calculate total cross section for given neutrino energy
   \item[b)] Using results from a), generate $\frac{d\sigma}{dT_e}$ distribution
   \item[c)] Using results from a), generate $\frac{d\sigma}{d\cos\theta}$ distribution, hint:  
  \end{itemize}

   $$T_e = \frac{2m_eE_\nu^2\cos^2\theta}{(m_e + E_\nu)^2 - E_\nu^2\cos^2\theta} \approx 2m_e \frac{\cos^2\theta}{1 - \cos^2\theta}$$

\vfill\null
\end{slide}