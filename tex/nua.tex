\section[toc=$\nu A$ interactions]{Neutrino interactions: nucleus}

\begin{slide}{Impulse approximation}
\null\vfill

  \twocolumn
  {
    \begin{itemize}
      \item In impulse approximation neutrino interacts with a single nucleon
      \item If $|\vec q|$ is low the impact area usually includes many nucleons
      \item For high $|\vec q|$ IA is justified
    \end{itemize}    
  }
  {
    \centering\scalebox{0.5}{\input{figures/ia}}
  }
  
  \begin{itemize}
    \item Squares of transition matrices are summed up and interference terms are neglected
  
    $$\sigma^A = \sum\limits_{i = 1}^Z \sigma_p + \sum\limits_{i = 1}^{A - Z}\sigma_n$$
    
    \item High $|\vec q|$ means more than 400~MeV. However, IA is always assumed
  \end{itemize}
  
\vfill\null
\end{slide}

\begin{slide}{Fermi gas}
\null\vfill

  \twocolumn
  {
    \myFrame{Nucleons move freely within the nuclear volume in constant binding potential.}
  }
  {
    \vspace*{-10pt}
    \input{figures/fermigas}
  }
  \twocolumn
  {
    \myBox{Global Fermi Gas}
    $$p_F = \frac{\hbar}{r_0}\left(\frac{9\pi N}{4A}\right)^{1/3}$$  
  }
  {
    \myBox{Local Fermi Gas}
    $$p_F(r) = \hbar\left(3\pi^2\rho(r) \frac{N}{A}\right)^{1/3}$$  
  }
  
  \centering\includegraphics[width=0.5\columnwidth]{figures/fermigas.eps}
  
\vfill\null
\end{slide}

\begin{wideslide}[toc=Spectral function]{Spectral function}
\null\vfill
    
  \twocolumn
  {
    \sep\sep
    \myFrame{The probability of removing of a nucleon with momentum $\vec p$ and leaving residual nucleus with excitation energy $E$.}
    $$P (\vec p, E) = P_{MF} (\vec p, E) + P_{corr} (\vec p, E)$$
    \rput[c](7.5, 2.0){\scalebox{0.75}{\input{figures/src}}}
  }
  {
    \vspace*{-10pt}
    \includegraphics[width=\columnwidth]{figures/sf_mom.eps}
    \includegraphics[width=\columnwidth]{figures/fg_vs_sf.eps}  
  }

\vfill\null
\end{wideslide}

\begin{slide}[toc=Two-body current]{Two-body current interactions}
\null\vfill

  \sep
  \twocolumn
  {
    \input{figures/mecNames}
  }
  {
    \rput[c](6.5, 1.6){\scalebox{0.75}{\input{figures/src2}}}
  }
  
  \vspace{-20pt}
  
  \myBoxFullWidth{Models in generators}
  
  \begin{itemize}
    \item Nieves model (GENIE - coming soon, NEUT, NuWro)
    \item Transverse Enhancement (TE) model by Bodek (NuWro)
    \item Dytman model (GENIE)
  \end{itemize}
  
\vfill\null
\end{slide}

\begin{wideslide}[toc=]{Two-body current interactions}
\null\vfill

  \begin{itemize}
    \item Nieves model is microscopic calculation
    \item TE model introduce $2p-2h$ contribution by modification of the vector magnetic form factors
  \end{itemize}
  
  \sep\sep

  \twocolumn
  {
    \myBox{Total MEC cross section}
    \includegraphics[width=\columnwidth]{figures/mec_xsec.eps}
  }
  {
    \myBox{MEC / (QEL + MEC)}
    \includegraphics[width=\columnwidth]{figures/mec_ratio.eps}  
  }


\vfill\null
\end{wideslide}

\begin{wideslide}[toc=]{Two-body current interactions}
\null\vfill

  \begin{itemize}
    \item Both models provide only the inclusive double differential cross section for the final state lepton
    \item Final nucleons momenta are set isotropically in CMS
  \end{itemize}

  \sep\sep
  
  \twocolumn
  {
    \myBox{Nieves}
    \includegraphics[width=\columnwidth]{figures/mec_lep_nieves.eps}
  }
  {
    \myBox{Transverse Enhancement}
    \includegraphics[width=\columnwidth]{figures/mec_lep_tem.eps}
  }


\vfill\null
\end{wideslide}

\begin{slide}[toc=COH pion production]{Coherent pion production}
\null\vfill

  \twocolumn
  {
    \begin{itemize}
      \item Rein-Sehgal model is commonly used for coherent pion production
      \item Note: it is different model than for RES
      \item Berger-Sehgal model replaces RS (NuWro, GENIE - coming soon)
    \end{itemize}
  }
  {
    \sep\sep
    \scalebox{0.75}{\input{figures/cohDiagram}}
  }
  
  \vspace*{-10pt}
  
  \myBoxFullWidth[pdcolor3]{Comments}
  
  \begin{itemize}
    \item In COH the residual nucleus is left in the same state (not excited)
    \item The interaction occurs on a whole nucleus - no final state interactions
  \end{itemize}

\vfill\null
\end{slide}

\begin{slide}[toc=Summary]{Neutrino interactions - summary}
\null\vfill

  \input{figures/neutrinoInteractions}

\vfill\null
\end{slide}