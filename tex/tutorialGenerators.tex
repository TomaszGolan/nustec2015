\section[toc=Tutorial generators]{Tutorial: analyzing MC output}

\begin{slide}{Introduction}
\null\vfill

  \begin{itemize}
   
    \item Each neutrino MC event generator performs simulations in two steps:
    
    \begin{itemize}
      \item cross section calculation
      \item event generation
    \end{itemize}
    
    \item Usually, two output files are produced:
    
    \begin{itemize}
     \item cross section per channel (sometimes as a function of energy - GENIE, sometimes integrated over flux - NuWro)
     \item ROOT file with TTree of events
    \end{itemize}
        
  \end{itemize}

\vfill\null
\end{slide}

\begin{slide}[toc=]{Introduction}
\null\vfill

  \begin{itemize}
  
    \item During tutorial we are going to work with GENIE's gst file (as it requires only ROOT)
    
    \item You can find it here: \href{http://home.fnal.gov/~goran/NUSTEC_tutorial/nustec_sample.gst.root}{[gst file]}
    
    \item If you are interested in more technical details on running generators visit the web page of NuSTEC school in Liverpool:
    
  \end{itemize}
  
  \sep
  
  \centering\href{http://school.genie-mc.org/}{NuSTEC Neutrino Generator School}

\vfill\null
\end{slide}

\begin{slide}{gst files}
\null\vfill

  \begin{itemize}
    \item The 'gst' is a GENIE summary ntuple format
    \item Selected branches:
    
    \begin{itemize}
      \item[\bf iev] number of events
      \item[\bf Q2] momentum transfer squared
      \item[\bf W] invariant mass
      \item[\bf nf] number of final state particles in hadronic system
      \item[\bf nfpip] number of final state $\pi^+$
      \item[\bf Ef(i)] energy of $i$-th particle in hadronic system
    \end{itemize}
    
    \item You can find all of them in GENIE manual (p. 112): \href{http://arxiv.org/pdf/1510.05494v1}{[GENIE manual]}
    
  \end{itemize}

\vfill\null
\end{slide}

\begin{slide}[method=direct]{Interactive ROOT}
\null\vfill

\begin{Verbatim}[commandchars=\\\{\}] 
\color{pdcolor3}\it # load ROOT file
root [0] TFile *myFile = new TFile ("file.gst.root")

\color{pdcolor3}\it # load TTree
root [1] TTree *myTree = myFile->Get("gst")

\color{pdcolor3}\it # lepton energy distribution
root [2] myTree->Draw("El")

\color{pdcolor3}\it # lepton energy distribution for events with 1 hadron in the final state
root [3] myTree->Draw("El", "nf == 1")

\color{pdcolor3}\it # use TBrowser if you prefer mouse over keyboard
root [4] TBrowser t
\end{Verbatim}

\vfill\null
\end{slide}

\begin{wideslide}[method=direct]{Script example}
\null\vfill
 
\begin{Verbatim}[commandchars=\\\{\}] 
void analyzeGST (const char *inputFile)
\{
  TFile *file = new TFile (inputFile);     \color{pdcolor3}\it// load root file
  TTree *tree = (TTree*)file->Get ("gst"); \color{pdcolor3}\it// get proper tree
  
  TH1D *leadingPip = new TH1D ("pip energy", "pip energy", 50, 0, 1);
  
  double leptonEnergy;      \color{pdcolor3}\it// lepton energy
  int nParticlesFS;         \color{pdcolor3}\it// number of particle in final state
  int nPip;                 \color{pdcolor3}\it// number of positive pion
  int hadronPDG[100];       \color{pdcolor3}\it// final state hadron pdg
  double hadronEnergy[100]; \color{pdcolor3}\it// final state hadron energy
  
  \color{pdcolor3}\it// set up branches
  tree->SetBranchAddress ("El",    &leptonEnergy);
  tree->SetBranchAddress ("nf",    &nParticlesFS);
  tree->SetBranchAddress ("nfpip", &nPip);
  tree->SetBranchAddress ("pdgf",  hadronPDG);
  tree->SetBranchAddress ("Ef",    hadronEnergy);  
  ...
\end{Verbatim}
 
\vfill\null
\end{wideslide}

\begin{wideslide}[method=direct, toc=]{Script example}
\null\vfill
 
\begin{Verbatim}[commandchars=\\\{\}]   
  ...
  const int nEvents = tree->GetEntries(); \color{pdcolor3}\it// get number of events

  for (int i = 0; i < nEvents; i++) \color{pdcolor3}\it// events loop
  \{
      tree->GetEntry (i); \color{pdcolor3}\it// get i-th event
      
      if (nPip == 0) continue;
      
      double maxEnergy = 0.0;
            
      for (int j = 0; j < nParticlesFS; j++) \color{pdcolor3}\it// particle loop
        if (hadronPDG[j] == 211 && hadronEnergy[j] > maxEnergy)
	  maxEnergy = hadronEnergy[j];        
            
      leadingPip->Fill (maxEnergy);
  \} \color{pdcolor3}\it// events loop
  
  leadingPip->Draw();
\}
\end{Verbatim}
 
\vfill\null
\end{wideslide}

\begin{slide}[toc=Task 1]{Task 1: basic informations}
\null\vfill

  \begin{itemize}
    \item Using interactive ROOT find the following information on the simulation:
    
    \sep
    
    \begin{itemize}
     \item number of events
     \item neutrino beam (flavor, energy)
     \item target
     \item dynamics
    \end{itemize}
  \end{itemize}

\vfill\null
\end{slide}

\begin{slide}[toc=Task 2]{Task 2: basic distributions}
\null\vfill

  \begin{itemize}
    \item Using interactive ROOT plot the following distributions:
    
    \sep
    
    \begin{itemize}
     \item proton energy (before FSI)
     \item proton energy (after FSI)
     \item lepton energy
     \item lepton energy for QEL
     \item lepton energy for all events with no meson in the final state
    \end{itemize}
  \end{itemize}

\vfill\null
\end{slide}

\begin{slide}[toc=Task 3]{Task 3: QEL}
\null\vfill

  \begin{itemize}
    \item Lets define QEL-like events as those without mesons in the final state
    \item On the same plot draw $\frac{d\sigma}{dQ^2}$ [in arbitrary units]:
    
    \sep
    
    \begin{itemize}
      \item total QEL-like
      \item contribution from true QEL
      \item background for QEL
    \end{itemize}

    \item What cut could you apply to reduce background?
    
  \end{itemize}

\vfill\null
\end{slide}

\begin{slide}[toc=Task 4]{Task 4: reconstructed energy}
\null\vfill

  \begin{itemize}
   
    \item For QEL neutrino scattering on a nucleon at rest the incoming neutrino energy can be expressed by lepton kinematics:

    $$E_\nu^{rec} = \frac{2(M_N - E_B)E_\mu - (E_B^2 - 2M_NE_B + m_\mu^2)}{2[M_N - E_B - E_\mu + |\vec k_\mu|\cos\theta_\mu]}$$
    
    \item Compare true and reconstructed neutrino energy for QEL events
    
    \item Compare true and reconstructed neutrino energy for QEL-like events

  \end{itemize}

\vfill\null
\end{slide}