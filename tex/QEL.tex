\section[toc=Quasi-elastic scattering]{Quasi-elastic scattering \\ \small{Building a generator step by step}}

\begin{wideslide}[toc=QEL on free N]{Quasi-elastic scattering on a free nucleon}
\null\vfill

  \myBox{Llewellyn-Smith formula}
  
  $$\frac{d\sigma}{d|q^2|} {{\nu_l + n \rightarrow l^- + p}\choose{\bar\nu_l + p \rightarrow l^+ + n}} = \frac{M^2G_F^2\cos\theta_C}{8\pi E_\nu^2}\left[A(q^2) \mp B(q^2)\frac{(s - u)}{M^2} + C(q^2)\frac{(s - u)^2}{M^4}\right]$$
  
  \myBox{Notation}
  
  \begin{itemize}
    \item Constants: $M$ - nucleon mass, $G_F$ - Fermi constant, $\theta_C$ - Cabibbo angle,
    \item $q^2 = (k - k')^2 = (p' - p)^2$ - four-momentum squared, where $k$, $k'$, $p$, $p'$ are four-momenta of initial and final lepton, initial and final nucleon
    \item $E_\nu$ - neutrino energy
    \item $s = (k + k')^2$ and $u = (k - p')^2$ - Mandelstam variables
  \end{itemize}

  
\vfill\null
\end{wideslide}

\begin{wideslide}[toc=]{Quasi-elastic scattering on a free nucleon}
\null\vfill

  \myBox{Llewellyn-Smith formula}
  
  $$\frac{d\sigma}{d|q^2|} {{\nu_l + n \rightarrow l^- + p}\choose{\bar\nu_l + p \rightarrow l^+ + n}} = \frac{M^2G_F^2\cos\theta_C}{8\pi E_\nu^2}\left[A(q^2) \mp B(q^2)\frac{(s - u)}{M^2} + C(q^2)\frac{(s - u)^2}{M^4}\right]$$
  
  \myBox{General idea}
  
  \begin{itemize}
    \item Having $k$ and $p$, generate $k'$ and $p'$
    \item Calculate $q^2$ and $(s - u) = 4ME_\nu + q^2 -m^2$ based on generated kinematics
    \item Calculate cross section
    \item Repeat $N$ times and the result is given by: 
    
    $$\sigma_{total} \sim \frac{1}{N} \sum\limits_{i = 1}^N \sigma (q_i^2)$$
    
  \end{itemize}

\vfill\null
\end{wideslide}

\begin{slide}{Generating kinematics}
\null\vfill

  \twocolumn
{
\centering
\begin{tikzpicture}

  \draw [notFilled=pdcolor3, thin] (-0.5, -0.5) -- (-0.5, 1.5) -- node[below, yshift = -0.25 cm] {\color{pdcolor1}LAB} (3.5, 1.5) -- (3.5, -0.5) -- (-0.5, -0.5);

  \draw [filled = pdcolor1] (0, 0) circle (0.25);
  \draw [filled = pdcolor6] (3, 0) circle (0.25);
  \draw [line, ultra thick, ->] (0.25, 0) -- node[above] {$\vec p_\nu$} (1.5, 0);
  \draw [line, ultra thick, ->, color = pdcolor6] (2.75, 0.1) -- node[above, xshift = 0.25cm] {$\vec p_N$} (2.0, 0.5);

\end{tikzpicture}
}
{
\centering
\begin{tikzpicture}

  \draw [notFilled=pdcolor3, thin] (-0.5, -0.5) -- (-0.5, 1.5) -- node[below, yshift = -0.25 cm] {\color{pdcolor1}CMS} (3.5, 1.5) -- (3.5, -0.5) -- (-0.5, -0.5);

  \draw [filled = pdcolor1] (0, 0) circle (0.25);
  \draw [filled = pdcolor6] (3, 0) circle (0.25);
  \draw [line, ultra thick, ->] (0.25, 0) -- node[above] {$\vec p*$} (1.25, 0);
  \draw [line, ultra thick, ->, color = pdcolor6] (2.75, 0) -- node[above] {$\vec p*$} (1.75, 0);

\end{tikzpicture}
}

  \begin{itemize}
    \item Lets consider kinematics in center-of-mass system
    \item Mandelstam $s$ is invariant under Lorentz transformation
   
    $$s = (k + p)^2 = (E + E_p)^2 - (\vec k + \vec p)^2 = (E^* + E_p^*)^2$$
   
    \item $\sqrt{s}$ is the total energy in CMS

    $$\sqrt{s} = E^* + E_p^* = \sqrt{p^{*2} + m^2} + \sqrt{p^{*^2} + M^2}$$
    
    \item We will use it to calculate $p*$
   
  \end{itemize}

\vfill\null
\end{slide}

\begin{slide}[toc=]{Generating kinematics}
\null\vfill
  
  \begin{itemize}
   \item Lets do some simple algebra:

  \vspace{-10pt}
  \begin{eqnarray*}
    \sqrt{s} & = & E^* + E_p^* = \sqrt{p^{*2} + m^2} + \sqrt{p^{*^2} + M^2} \\
    \sqrt{s} & = & E^* + \sqrt{E^{*2} - m^2 + M^2} \\
    s & = & E^{*2} + E^{*2} - m^2 + M^2 + 2E^*E_p^* \\
    s & = & 2E^*(E^* + E_p^*) - m^2 + M^2 \\
    s & = & 2E^*\sqrt{s} - m^2 + M^2 \\
    E^* & = & \frac{s + m^2 - M^2}{2\sqrt{s}} \\
    E_p^* & =&  \frac{s + M^2 - m^2}{2\sqrt{s}} \mbox{ (analogously)}
  \end{eqnarray*}
  
  \item After more algebra we get:
  
  \vspace{-10pt}
  $$p^* = \sqrt{E^{*2} - m^2} = \frac{[s - (m - M)^2]\cdot[s - (m + M)^2]}{2\sqrt{s}}$$

  \end{itemize}

\vfill\null
\end{slide}

\begin{slide}[toc=]{Generating kinematics}
\null\vfill

  \twocolumn
  {
    \sep
    \begin{itemize}
      \item We use spherical coordinate system to determine momentum direction in CMS:
    \end{itemize}
    $$\vec p^* = p^* \cdot (\sin\theta\cos\phi,~~\sin\theta\sin\phi,~~\cos\theta)$$
  }
  {
    \centering\tdplotsetmaincoords{60}{110}

\pgfmathsetmacro{\rvec}{.8}
\pgfmathsetmacro{\thetavec}{45}
\pgfmathsetmacro{\phivec}{60}

\begin{tikzpicture}[scale = 2, tdplot_main_coords]

  \coordinate (O) at (0,0,0);
  \draw[thick, >=latex, ->] (0,0,0) -- (1,0,0) node[anchor=north east]{$x$};
  \draw[thick, >=latex, ->] (0,0,0) -- (0,1,0) node[anchor=north west]{$y$};
  \draw[thick, >=latex, ->] (0,0,0) -- (0,0,1) node[anchor=south]{$z$};
  
  \tdplotsetcoord{P}{\rvec}{\thetavec}{\phivec}
  
  \draw[-stealth, color = pdcolor6] (O) -- (P) node[above right] {$p^*$};
  \draw[dashed, color = pdcolor6] (O) -- (Pxy);
  \draw[dashed, color = pdcolor6] (P) -- (Pxy);

  \tdplotdrawarc[color = pdcolor3]{(O)}{0.2}{0}{\phivec}{anchor=north}{$\phi$}

  \tdplotsetthetaplanecoords{\phivec}
  
  \tdplotdrawarc[tdplot_rotated_coords, color = pdcolor3]{(0,0,0)}{0.5}{0}{\thetavec}{anchor=south west}{$\theta$}

\end{tikzpicture}
  }  

  \begin{itemize}
    \item Generate random angles:  
    \item[]
    
    \begin{tabular}{rclll}
           $\phi$ & $ = $ &  $2\pi\cdot\mbox{random}[0,1]$ & $\Rightarrow$ & $\sin\phi,~~\cos\phi$ \\ \\
     $\cos\theta$ & $ = $ & $2\cdot\mbox{random}[0,1] - 1$ & $\Rightarrow$ & $\sin\theta, ~~\cos\theta$  
    \end{tabular}
    
    \item[]
    
    \item All we need to do is to go back to LAB frame

  \end{itemize}

\vfill\null
\end{slide}

\begin{slide}{LAB $\leftrightarrows$ CMS}
\null\vfill

  \begin{itemize}
    
    \item Lorentz boost in direction $\hat n = \frac{\vec v}{v}$ of $(t,\vec r)$:
    
    \begin{eqnarray*}
      t' & = & \gamma \left(t - v \hat n\cdot\vec r\right) \\
      \vec r' & = & \vec r + (\gamma - 1)(\hat n\cdot\vec r)\hat n - \gamma t v \hat n
    \end{eqnarray*}
    
  \end{itemize}
  
  \sep
  
  \twocolumn
  {
    \begin{itemize}
      
      \item In our case
  
      $$\vec v = \frac{\vec p_\nu + \vec p_N}{E_\nu + E_N}$$
    
      \item Boost from LAB to CMS in $\vec v$ direction
    
      \item Boost from CMS to LAB in $-\vec v$ direction
    
    \end{itemize}
  }
  {
    \begin{tikzpicture}

  \draw [notFilled=pdcolor3, thin] (-0.5, -0.5) -- (-0.5, 1.5) -- node[below, yshift = -0.25 cm] {\color{pdcolor1}LAB} (3.5, 1.5) -- (3.5, -0.5) -- (-0.5, -0.5);

  \draw [filled = pdcolor1] (0, 0) circle (0.25);
  \draw [filled = pdcolor6] (3, 0) circle (0.25);
  \draw [line, ultra thick, ->] (0.25, 0) -- node[above] {$\vec p_\nu$} (1.5, 0);
  \draw [line, ultra thick, ->, color = pdcolor6] (2.75, 0.1) -- node[above, xshift = 0.25cm] {$\vec p_N$} (2.0, 0.5);

\end{tikzpicture}

\sep

\begin{tikzpicture}

  \draw [notFilled=pdcolor3, thin] (-0.5, -0.5) -- (-0.5, 1.5) -- node[below, yshift = -0.25 cm] {\color{pdcolor1}CMS} (3.5, 1.5) -- (3.5, -0.5) -- (-0.5, -0.5);

  \draw [filled = pdcolor1] (0, 0) circle (0.25);
  \draw [filled = pdcolor6] (3, 0) circle (0.25);
  \draw [line, ultra thick, ->] (0.25, 0) -- node[above] {$\vec p*$} (1.25, 0);
  \draw [line, ultra thick, ->, color = pdcolor6] (2.75, 0) -- node[above] {$\vec p*$} (1.75, 0);

\end{tikzpicture}

  }
  
\vfill\null
\end{slide}

\begin{wideslide}[toc=Cross section]{Calculating cross section}
\null\vfill

  \myBox{Llewellyn-Smith formula}
  
  $$\frac{d\sigma}{d|q^2|} {{\nu_l + n \rightarrow l^- + p}\choose{\bar\nu_l + p \rightarrow l^+ + n}} = \frac{M^2G_F^2\cos\theta_C}{8\pi E_\nu^2}\left[A(q^2) \mp B(q^2)\frac{(s - u)}{M^2} + C(q^2)\frac{(s - u)^2}{M^4}\right]$$
  
  \myBox{Calculation}

  \begin{itemize}
    \item Once we have $p'$ and $k'$ in LAB frame we can calculate $q^2$ and $(s - u)$
    \item Once we have $q^2$ we can calculate $A(q^2)$, $B(q^2)$, $C(q^2)$
    \item We have everything to calculate cross section
    \item Do we? Or maybe we are still missing something?
  \end{itemize}

\vfill\null
\end{wideslide}

\begin{wideslide}[toc=]{Calculating cross section}
\null\vfill

  \myBox{Llewellyn-Smith formula}
  
  $$\frac{d\sigma}{d|q^2|} {{\nu_l + n \rightarrow l^- + p}\choose{\bar\nu_l + p \rightarrow l^+ + n}} = \frac{M^2G_F^2\cos\theta_C}{8\pi E_\nu^2}\left[A(q^2) \mp B(q^2)\frac{(s - u)}{M^2} + C(q^2)\frac{(s - u)^2}{M^4}\right]$$
  
  \myBox{Calculation}

  \begin{itemize}
    \item Once we have $p'$ and $k'$ in LAB frame we can calculate $q^2$ and $(s - u)$
    \item Once we have $q^2$ we can calculate $A(q^2)$, $B(q^2)$, $C(q^2)$
    \item We have everything to calculate cross section
    \item Do we? Or maybe we are still missing something?
  \end{itemize}
  \sep
  \centering{\color{pdcolor6} We change the variable we integrate over! We need Jacobian!}

\vfill\null
\end{wideslide}

\begin{wideslide}[toc=]{Calculating cross section}
\null\vfill

  \begin{itemize}
    
    \item Express $q^2$ in terms of angle:
    
    $$q^2 = (k - k')^2 = m^2 - 2kk' = m^2 - 2EE' + 2|\vec k||\vec k'|\cos\theta $$
    
    \item Thus, the Jacobian is given by:
    
    $$dq^2 = 2|\vec k||\vec k'|d(\cos\theta)$$
    
    {\it\color{pdcolor3}Note: must be calculated in CMS}
    
    \item Finally, our total cross section is given by:
    
    \begin{eqnarray*}
      \sigma      & = & \int\limits_{-1}^{1} \frac{M^2G_F^2\cos\theta_C}{8\pi E_\nu^2}\left[A(q^2) \mp B(q^2)\frac{(s - u)}{M^2} + C(q^2)\frac{(s - u)^2}{M^4}\right]2|\vec k||\vec k'|d\cos\theta \\
      \sigma_{MC} & = & \frac{2}{N}\sum\limits_{i = 1}^N \frac{M^2G_F^2\cos\theta_C}{8\pi E_\nu^2}\left[A(q^2) \mp B(q^2)\frac{(s - u)}{M^2} + C(q^2)\frac{(s - u)^2}{M^4}\right]2|\vec k||\vec k'|
    \end{eqnarray*}

  \end{itemize}

\vfill\null
\end{wideslide}

\begin{slide}{Optimization}
\null\vfill

  \twocolumn
  {
    \sep
    \begin{itemize}
      \item We want to avoid any sharp peaks
      \item They affect our efficiency
      \item They may also affect the accuracy
      \item Lets change variable once again:
      
      $$\cos\theta = 1 - 2u^2$$
     
      where $u\in[0,1]$
      
      \item Note extra Jacobian and new integration limits
      
    \end{itemize}
  }
  {
    \usetikzlibrary{calc}

\pgfplotsset{every  tick/.style={pdcolor3,}, minor x tick num=1,}

\begin{tikzpicture}

  \begin{axis}[xlabel = {$\cos\theta$}, ylabel = {$\frac{d\sigma}{d\cos\theta}$ [arbitrary units]}, ylabel near ticks, domain=-1:1, scale=0.5, axis lines = left, inner axis line style={>=latex}, ymin = 0, ymax = 1.2, xmin = -1, xmax = 1.2]
    
    \addplot [thick, color = pdcolor6] table[x = x, y = y, col sep = space, mark = none] {data/qelKinCos.dat};
    
  \end{axis}

\end{tikzpicture}
    \vspace{-10pt}
    \usetikzlibrary{calc}

\pgfplotsset{every  tick/.style={pdcolor3,}, minor x tick num=1,}

\begin{tikzpicture}

  \begin{axis}[xlabel = {$x$}, ylabel = {$\frac{d\sigma}{dx}$ [arbitrary units]}, ylabel near ticks, domain=0:1, scale=0.5, axis lines = left, inner axis line style={>=latex}, ymin = 0, ymax = 1.2, xmin = 0, xmax = 1.2]
    
    \addplot [thick, color = pdcolor6] table[x = x, y = y, col sep = space, mark = none] {data/qelKinX.dat};
    
  \end{axis}

\end{tikzpicture}
  }
  
\vfill\null
\end{slide}
